\section{Introduction}
In our society today, a surefire way of making profit is by developing methods and tools that promise to help us manage life in the fast lane. One of the older methods with this purpose that is gaining in popularity again is the rapid intake of written information, also known as speed reading.
\subsection{Speed Reading} 
Interest in increasing one's reading rate was sparked when \citeauthor{javal1879essai} drew attention to eye movements during reading through a series of articles published in the 1870s \citep{javal1879essai, huey1908psychology, wade2009did}. Rather than reading words letter by letter, reading was now seen as an alternating process of fixations and saccades. Each fixation allows good readers to take in on average approximately three words. During saccades, nothing can be seen \citep{ahuja1995increase}. Research was fueled further after the American psychologist \citeauthor{renshaw1945visual} employed tachistoscopes to train pilots of the U.S. Navy in the second World War \citep{godnig2003tachi, benschop1998tachistoscope, renshaw1945visual}. A tachistoscope is an apparatus that allows the rapid serial visual presentation (RSVP) of stimuli. With this method, saccades could be eliminated by presenting words in the same location. 
%"Interest in the conditioning or control of eye movement resulted in the making of many other tachistoscopic devices such as the \textit{Reading Accelerator}, the \textit{Reading Rate Controller}, the \textit{Fashmeter}, and the \textit{Rate Reader}"\citep[p.~104]{witty1969rate}. 
The research regarding the effectiveness of RSVP to increase reading speed produced converging evidence that comprehension suffers under this approach \citep{swalm1973speed, witty1969rate, causey1954colleges, bormuth1961tachistoscope, mcconkie1973experimental, schotter2014don}. Despite such evidence, using RSVP for speed reading has recently resurfaced in popular culture in the form of smartphone and wearable applications, such as the one by Spritz \footnote{http://www.spritzinc.com/}. Several studies have shown the importance of reading in first and second language acquisition, with reading improving not only comprehension but also inducing an intuition of the underlying syntactic structure \citep{chomsky1972stages, siegel1988development,krashen1993power}. Thus, we are not quite ready to write off RSVP as a method for speed reading yet. In this project, we study the possibilities of the method in artificial grammar learning (AGL).

\subsection{Artificial Grammar Learning}
In the 1960s, \citeauthor{reber1967implicit} studied the implicit learning of stimulus structure. By conducting a series of experiments, he defined three requirements \citep[p.~190]{reber1978analogic} for such learning to occur:
\begin{enumerate}
\item The rules governing the stimulus material must be complex.
\item Participants should pay close attention to stimuli without searching for underlying rules.
\item Participants should be unable to verbalize the knowledge they obtained during learning. 
\end{enumerate}
\citeauthor{reber1967implicit} employed artificial grammars to ensure meeting the first requirement. Artificial grammars are finite state machines with the transition from one state to the next corresponding to one element of the grammar. Grammatical patterns, or words, are created by making only allowed transitions from the start state to the final state (see Figure\ref{figure1}). 
\begin{figure}
\centering
\begin{tikzpicture}[shorten >=1pt,node distance=2cm,on grid,auto] 
   \node[state,initial] (S_0)   {$S_0$}; 
   \node[state] (S_1) [above right=of S_0] {$S_1$}; 
   \node[state] (S_2) [below right=of S_0] {$S_2$}; 
   \node[state,accepting] (S_3) [right=of S_1] {$S_3$}; 
   \node[state,accepting] (S_4) [right=of S_2] {$S_4$}; 
   \node[state,accepting](S_5) [below right=of S_3] {$S_5$};
    \path[->] 
    (S_0) edge  node {M} (S_1)
          edge  node [swap] {V} (S_2)
    (S_1) edge  node  {V} (S_3)
          edge [loop above] node {S} ()
    (S_2) edge  node {X} (S_4)
          edge  node {X} (S_1)
    (S_3) edge  node  {R} (S_2)
          edge  node  {S} (S_5)
    (S_4) edge  node  {M} (S_5)
          edge [loop below] node {R} ();
\end{tikzpicture}
\caption{An artificial grammar is given by a finite state machine. States are indicated by circles labeled $S_{i}$. $S_{0}$ is the initial state. Every grammatical item has to start with a transition via an arrow from the initial state to a state directly connected to it (in this case $S_{1}$ or $S_{2}$). States encircled by two lines are final states ($S_{3}$, $S_{4}$, $S_{5}$). The machine can be exited from any final state. The arrows indicate legal transition directions (in this example, it is legal to move from $S_{3}$ to $S_{2}$ but not the other way around). Words are formed by concatenating the letters assigned to each arrow in the order of making the transition (grammatical words are for example MVS, VXRRR).} 
\label{figure1}
\end{figure}
%Insert some figure of an AG with a nice caption explaining how words are formed
The AGL paradigm consists of a learning and a testing phase. In the learning phase, participants are confronted with a set of grammatical words\footnote{In his initial studies \citep{reber1967implicit}, Reber asked participants to memorize these but later went on to study learning by mere exposure \citep{reber1978analogic}.}. In the testing phase, acquired knowledge is tested by presenting new words and asking participants to judge their grammaticality. 
This raises the question: How is it possible to develop a feeling for apparently nonsensical patterns? One theory is that implicit learning is statistical learning. This means that the grammatical patterns in the training phase are used to train a classifier within the brain, which tells us the probability for a new stimulus to be a word of the grammar or not \citep{perruchet2006implicit,saffran2002constraints}.
While \citeauthor{reber1967implicit} exposed participants to a total of 20 exemplars three times and each exemplar was regarded for ten seconds \citep{reber1978analogic}, later studies showed similar results for shorter exposure \citep{knowlton1996artificial,gomez1999artificial}\footnote{\citeauthor{knowlton1996artificial} exposed participants to stimuli for three seconds per item and then asked them to reproduce the item. When participants could not get it correct on this reproduction attempt, this could be repeated twice, so that theoretically participants could have seen items for a total of nine seconds. However, it is mentioned that few participants needed a third exposure to reproduce the item correctly. \citeauthor{gomez1999artificial} exposed 1 year-old toddlers to auditory exemplars of an artificial grammar that included vowels for a total of two minutes.}.\\
In this project, we are interested in the possibility of implicitly learning an artificial grammar by viewing exemplars presented in an RSVP paradigm. Specifically, we examine two research questions:
\begin{enumerate}
\item \label{item1}Does AGL occur when words are presented individually at the same location at a normal reading rate?
\item \label{item2}Does it also occur at faster presentation speeds (as used by speed reading tools) when
\begin{enumerate}
\item \label{item2a}the same amount of material is presented?
\item \label{item2b}material is presented for the same amount of time? 
\end{enumerate}
\end{enumerate}
To our knowledge, the limitations of AGL have not yet been studied with regard to the presentation speed of exemplars. We generally regard AGL as possible when exemplars are presented at normal and rapid speeds. This is due to the high attentional demand of RSVP (REFERENCE NEEDED) in combination with the importance of attention in implicit learning. However, if implicit learning is statistical learning, we further believe that longer exposure times will most likely lead to more reliable probabilistic inferences, with a repeated exposure potentially making up for a shorter exposure. Therefore, we expect an affirmative answer to research questions \ref{item1} and \ref{item2b}. 


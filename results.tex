\section{Results}
The total number of correct classifications on the 100 test items, $P_{C}$, was determined for each participant. To compare observed to chance level performance, the Wilcoxon signed-rank test was employed for the \textit{normal} and the \textit{fast-time} conditions, while a one-sample t-test could be used for the \textit{fast-amount} condition\footnote{The Shapiro-Wilk test was used to check whether it could be assumed that the samples were drawn from a normally distributed population for each condition. We chose this test because it has the greatest statistical power for small sample sizes \citep{yazici2007comparison}}. The tests were performed one-tailed because only above chance level performance was of interest. As can be seen from the results of these tests, illustrated in Table\ref{table1}, no group performed significantly above chance. Therefore, it would not be sensible to make any statistical between-group comparisons. \\
\begin{table}[h]
\centering
\begin{tabular}{l l l l }
\toprule
\textbf{Group} & \textbf{Chance Level Comparison} & \textbf{\textit{p}-value} \\
\midrule
  normal      & Wilcoxon Signed Rank test        & .068 \\
  fast-amount & One-sample \textit{t}-test       & .057 \\
  fast-time   & Wilcoxon Signed Rank test        & .237 \\      
\bottomrule
\end{tabular}
\caption{This illustrates the analysis of $P_C$ scores. Each score was compared to chance level performance (i.e. 50\% correct and 50\% incorrect classifications). All tests were performed right-tailed, since below chance level performance was not relevant. None of the groups show significant implicit learning effects at the $\alpha=.05$ level.}
\label{table1}
\end{table}
Taking a closer look at the different types of test responses is not more conclusive. The 100 test items consisted of 50 unique items which were all repeated once. Thus, participants could classify each unique item in one of four ways:
\begin{enumerate}
\item Correct-Correct (CC): classified correctly on both presentations
\item Correct-Erroneous (CE): classified correctly on the first presentation but incorrectly on the second
\item Erroneous-Correct (EC): classified incorrectly on the first presentation but correctly on the second
\item Erroneous-Erroneous (EE): misclassified on both presentations
\end{enumerate}
The sum of CC and EE denotes overall consistency. Importantly, `when the status of the item is known, it is always classified correctly; when it is not known, a guess is made.' \citep[p.~227]{reber1989implicit}. Using this simple model as a basis, \citeauthor{reber1989implicit} states that CE, EC, and EE should not be statistically distinguishable from each other, since they all reflect guesses. On the other hand, if EE is significantly greater than the average of CE and EC, it can be inferred that judgments were based on rules that are not representative of the grammar. Furthermore, CC should be significantly higher than each of the other three variables if the participants actually implicitly learned a correct albeit partial representation of the grammar. Finally, if the difference between CE and EC or EC and CE is significant, it is indicative of forgetting or learning during the testing phase respectively (for an in-depth discussion see \citet{reber1989implicit}). For each group, Table\ref{table2} shows the means for the four variables. Intuitively, the illustrated behavior of the participants seems representative of guessing on all test items. Paired t-tests for the various combinations of these variables (all are distributed normally within each group) confirms this intuition. EC and CE do not show a significant difference. Thus, there is neither evidence for learning nor for forgetting during the testing phase. When comparing EE to the average of CE and EC, $t(9) \leq -4.216$ and $p < .01$ was obtained for all groups. Furthermore, CC was only significantly higher than EE in the \textit{normal} group ($t(9)=1.843$, $p=.049$). Taken together, this implies that all participants based a considerable amount of their decisions on rules that were not part of the grammar. This is in line with the responses of participants to the request `Please specify how you decided whether a string was ruleful or unruleful.' When rules were states on these forms, they were almost exclusively not representative of the grammar. 
\begin{table}[h]
\centering
\begin{tabular}{lllllll}
\toprule
& \multicolumn{6}{c}{\textbf{Group}} \\
\cmidrule{2-7}
\textbf{Parameter} & \multicolumn{2}{l}{\textbf{normal}} & \multicolumn{2}{l}{\textbf{fast-amount}} & \multicolumn{2}{l}{\textbf{fast-time}} \\
\midrule
 	    & Mean                       & SD                              & Mean & SD  & Mean & SD \\
\cmidrule(lr{.75em}){2-3}\cmidrule(lr{.75em}){4-5}\cmidrule(lr{.75em}){6-7}
$P_{C}$ & 53.3                       & 5.8                             & 52.9 & 5.2 & 50.9 & 4.8 \\
CC	    & 37.6                       & 5.8                             & 41.6 & 5.7 & 35.4 & 7.4 \\
CE 	    & 16.2                       & 6.8                             & 12.4 & 4.1 & 14.8 & 2.4 \\
EC	    & 15.4                       & 3.9                             & 10.2 & 5.7 & 16.2 & 1.9 \\
EE	    & 30.8                       & 7.5                             & 35.8 & 7.7 & 33.6 & 8.2 \\
consistency & 68.4                   & 6.2                             & 77.4 & 8.2 & 50.9 & 11.6\\
\bottomrule
\end{tabular}
\caption{This shows the mean and SD for each group for each of the six classification variables. The test consisted of 100 items: 50 unique items presented twice. $P_C$ denotes the total number of correctly classified items. Since each item was presented twice, it could be classified correctly twice (CC), correctly on the first presentation but not on the second (CE), correctly on the second but not on the first presentation (EC), correctly on neither presentation (EE). The consistency is the sum of CC and EE.}
\label{table2}
\end{table}
Lastly, we correlated the number of languages participants speak as well as their prior speed reading experience with their $P_C$-scores across groups. Slightly negative but non-significant correlations were obtained: $r=-.30$, $p=.11$ for the number of languages and $r=-.23$, $p=.22$ for the speed reading experience.